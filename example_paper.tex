%%%%%%%% ICML 2025 EXAMPLE LATEX SUBMISSION FILE %%%%%%%%%%%%%%%%%

\documentclass{article}

% Recommended, but optional, packages for figures and better typesetting:
\usepackage{microtype}
\usepackage{graphicx}
\usepackage{subfigure}
\usepackage{booktabs} % for professional tables
\usepackage{url}
\usepackage[most]{tcolorbox}
\usepackage{wrapfig}
\usepackage{enumitem} %for leftmargin
\usepackage{tabularx} % For adjustable-width columns
\usepackage{caption}
\captionsetup{font=small,labelfont=bf,skip=5pt}
\captionsetup[figure]{aboveskip=5pt,belowskip=-10pt}
\usepackage{longtable}
\usepackage{tabularx}
\usepackage{ltablex}
\usepackage{multirow}
\usepackage{mathtools}
\usepackage[table]{xcolor}
\usepackage{makecell}
\usepackage{algorithm}
\usepackage{algpseudocode}
\usepackage{bbm}
\usepackage{amsmath,amsthm,amssymb}
\usepackage{framed}

% hyperref makes hyperlinks in the resulting PDF.
% If your build breaks (sometimes temporarily if a hyperlink spans a page)
% please comment out the following usepackage line and replace
% \usepackage{icml2025} with \usepackage[nohyperref]{icml2025} above.
\usepackage{hyperref}


% Attempt to make hyperref and algorithmic work together better:

\newcommand{\authorcomment}[3]{%
  % \textcolor{#2}{[\textbf{#1}: #3]}%  % Disabled for submission
}
\newcommand{\yu}[1]{\textcolor{teal}{[Yu: #1]}} % Disabled for submission
\newcommand{\dr}[1]{\textcolor{blue}{DR: #1}}
%


% Use the following line for the initial blind version submitted for review:
% \usepackage{icml2025}

% If accepted, instead use the following line for the camera-ready submission:
\usepackage[accepted]{icml2025}

% For theorems and such (already loaded above)
% \usepackage{cleveref}

% if you use cleveref..
\usepackage[capitalize,noabbrev]{cleveref}

%%%%%%%%%%%%%%%%%%%%%%%%%%%%%%%%
% THEOREMS
%%%%%%%%%%%%%%%%%%%%%%%%%%%%%%%%
\theoremstyle{plain}
\newtheorem{theorem}{Theorem}[section]
\newtheorem{proposition}[theorem]{Proposition}
\newtheorem{lemma}{Lemma}[theorem]
\newtheorem{corollary}{Corollary}[theorem]

\theoremstyle{definition}
\newtheorem{definition}[theorem]{Definition}
\newtheorem{assumption}[theorem]{Assumption}
\theoremstyle{remark}
\newtheorem{remark}[theorem]{Remark}

% Todonotes is useful during development; simply uncomment the next line
%    and comment out the line below the next line to turn off comments
\usepackage[disable,textsize=tiny]{todonotes}
%\usepackage[textsize=tiny]{todonotes}


% The \icmltitle you define below is probably too long as a header.
% Therefore, a short form for the running title is supplied here:
\icmltitlerunning{An Explanation In Support Of Neuro-Symbolic Language Models for Scaling Algorithmic Reasoning}

\newcommand{\mc}{\mathcal}
\newcommand{\mb}{\mathbb}
\newcommand{\e}{\epsilon}
\newcommand{\g}{\Gamma}
\newcommand{\del}{\delta}
\newcommand{\li}[1]{\lim\limits_{#1 \to \infty}} %limit to infinity
\newcommand{\lz}[1]{\lim\limits_{#1 \to 0}} %limit to zero
\newcommand{\s}[2]{\sum\limits_{#1=1}^{#2}} %sum
\newcommand{\p}[2]{\prod\limits_{#1=1}^{#2}} %sum

\newcommand{\Hc}{\mathcal{H}} %hypothesis class
\newcommand{\N}{\mathbb{N}_+} %positive naturals
\newcommand{\Rd}{\mathbb{R}} %real numbers domain


\newcommand{\Sig}{|\Sigma|} %alphabet

% Space-saving formatting (ICML compliant)
\setlength{\textfloatsep}{3pt plus 1pt minus 1pt}
\setlength{\floatsep}{3pt plus 1pt minus 1pt}
\setlength{\intextsep}{3pt plus 1pt minus 1pt}
\setlength{\abovecaptionskip}{2pt}
\setlength{\belowcaptionskip}{-12pt}
\setlist{nosep,leftmargin=*,topsep=0pt,partopsep=0pt}
\setlength{\abovedisplayskip}{3pt}
\setlength{\belowdisplayskip}{3pt}
\setlength{\parskip}{0pt}
\setlength{\dbltextfloatsep}{3pt plus 1pt minus 1pt}
\setlength{\dblfloatsep}{3pt plus 1pt minus 1pt}

\begin{document}

\twocolumn[
\icmltitle{An Explanation In Support Of Neuro-Symbolic Language Models for Scaling Algorithmic Reasoning}

% It is OKAY to include author information, even for blind
% submissions: the style file will automatically remove it for you
% unless you've provided the [accepted] option to the icml2025
% package.

% List of affiliations: The first argument should be a (short)
% identifier you will use later to specify author affiliations
% Academic affiliations should list Department, University, City, Region, Country
% Industry affiliations should list Company, City, Region, Country

% You can specify symbols, otherwise they are numbered in order.
% Ideally, you should not use this facility. Affiliations will be numbered
% in order of appearance and this is the preferred way.
\icmlsetsymbol{equal}{*}

\begin{icmlauthorlist}
\icmlauthor{Terry Tong}{yyy}
\icmlauthor{Yu Feng}{yyy}
\icmlauthor{Surbhi Goel}{yyy}
\icmlauthor{Dan Roth}{yyy}
% \icmlauthor{Firstname5 Lastname5}{yyy}
% \icmlauthor{Firstname6 Lastname6}{sch,yyy,comp}
% \icmlauthor{Firstname7 Lastname7}{comp}
%\icmlauthor{}{sch}
% \icmlauthor{Firstname8 Lastname8}{sch}
% \icmlauthor{Firstname8 Lastname8}{yyy,comp}
%\icmlauthor{}{sch}
%\icmlauthor{}{sch}
\end{icmlauthorlist}

\icmlaffiliation{yyy}{University of Pennsylvania}
% \icmlaffiliation{comp}{Company Name, Location, Country}
% \icmlaffiliation{sch}{School of ZZZ, Institute of WWW, Location, Country}

\icmlcorrespondingauthor{Terry Tong}{tongt1@seas.upenn.edu}
% \icmlcorrespondingauthor{Firstname2 Lastname2}{first2.last2@www.uk}

% You may provide any keywords that you
% find helpful for describing your paper; these are used to populate
% the "keywords" metadata in the PDF but will not be shown in the document
\icmlkeywords{Neuro-symbolic AI, Tool Use, Bayesian Inference, Algorithmic Reasoning, Code Generation}

\vskip 0.3in
]


% this must go after the closing bracket ] following \twocolumn[ ...

% This command actually creates the footnote in the first column
% listing the affiliations and the copyright notice.
% The command takes one argument, which is text to display at the start of the footnote.
% The \icmlEqualContribution command is standard text for equal contribution.
% Remove it (just {}) if you do not need this facility.

%\printAffiliationsAndNotice{}  % leave blank if no need to mention equal contribution
\printAffiliationsAndNotice{\icmlEqualContribution} % otherwise use the standard text.




\begin{abstract}
% Context: Establish the problem domain
Large language models can solve algorithmic problems either through direct natural language (NL) reasoning or by generating executable code delegated to an external solver. However, little theoretical progress has been made on explaining \emph{why} code-based approaches consistently outperform natural language reasoning. Comparing NL reasoning and solver-based pipelines directly is ill-posed: they differ simultaneously in representation space and execution mechanism.
% Gap + Contribution: What's missing and what we provide
We introduce a three-arm framework that makes this comparison tractable by introducing an intermediary step---code generation with LLM-based execution. This enables our empirical analysis which shows a statistically significant gap supporting code $>$ NL across 44 different algorithmic tasks and 6 models. Empirically, we demonstrate that for evaluated models, NL reasoning does not contain decision-relevant information beyond what is already in code. Building on these observations, we introduce a statistical and information theoretic analysis that proves code $>$ NL under mild assumptions. 

% These results inform the design of compositional AI systems, providing principled guidance on when to use tool-augmented versus monolithic reasoning for algorithmic tasks. Our framework offers a unified perspective on the tool-use versus direct-reasoning tradeoff. 


% \dr{Say something about the outcome of this analysis; you jump directly to experiments leaving the reader expecting something.}

% Results: Concrete findings
% Across 44 different algorithmic tasks and 6 models, we observe that there is statistically significant gap (p$<$0.001) between code and natural language reasoning ($>$25\%). 
% \dr{The second part of the abstract is too long. You don't need to say everything. Say that the theoretical framework is supported by extensive experimentation (give the first sentence above) and summarize the rest in 1-2 sentences or drop it, since you want to keep the last, impact, sentence.} 

% We find that code representations scale better, with 4.01$\times$ odds of correctness compared to natural language. Under causal intervention experiments, we identify natural language reasoning as a projection of deeper underlying algorithmic representations. Using this insight, we leverage Bayesian Inference and the Blackwell Dominance Principle to prove that the code execution route achieves lower Bayes Risk than natural language reasoning route.
% Impact: Why it matters

\end{abstract}


\section{Introduction} \label{sec:intro}

Many agentic systems orchestrate symbolic solvers, LLMs and other tools, and demonstrate state-of-the-art performance ~\cite{gao2023pal,yao2023react, 10.5555/3666122.3669119, yang2024sweagent,wang2025mcpbenchbenchmarkingtoolusingllm}. Prior research empirically demonstrates that translating problems into solver-executable code representations (hereafter coined Route 2) and delegating execution often outperforms reasoning end-to-end in NL (hereafter coined Route 1) \yu{I am a bit confused what is hereafter coined Route 1, also you should link figure 1 in the introduction} on logic- and algorithmic-style complex reasoning tasks ~\cite{lyu2023faithful, pan2023logic}. However, it is unclear whether benefits come from the more structured code representation itself used in the planning and reasoning process, or the reliability of external solvers, or both. Little progress has been made in this direction since the problem itself is ill-posed: two routes learn fundamentally different objects, so there is no common formal target and metric to compare. 

\begin{figure}[t]
    \centering
    \vspace{-5pt}
    \includegraphics[width=\linewidth]{images/CleanShot 2026-01-28 at 08.37.18@2x.png}
    \vspace{-25pt}
    \caption{\textbf{Three-Route Framework.} We decompose algorithmic reasoning into: (1) \emph{Translation} into NL or code, and (2) \emph{Execution} via LLM or solver. This yields three routes: \textbf{Route~1} (Direct NL), \textbf{Route~2} (Code + NL simulation), \textbf{Route~3} (Code + Solver Execution). Prior work compares only Route~1 vs.\ Route~3, confounding translation and execution. Our Route~2 isolates these factors. }
    \label{fig:three_routes}
\end{figure}

This paper presents a systematic three-route framework (\cref{sec:framework}) breaks down the question into tractable pairwise comparisons between the different routes. We decouple the end-to-end pipeline into a trace generation phase (e.g. Chain of Thought \cite{wei2022chain}) and an execution phase (see \cref{def:trace}). This allows us to fix one phase, and make causal claims on the effects of phase on end-to-end reasoning performance. We first instantiate this framework to verify the overall ordering Route 1 (NL + NL Reasoning) $\leq$ Route 2 (Code + NL Reasoning) $<$ Route 3 (Code +  Python Execution) holds (\cref{sec:empirical}). Solver-based execution results (+29.9\% v.s. NL reasoning)  in best performance across different models \texttt{(Mistral}~\cite{jiang2023mistral}, \texttt{OpenAI}~\cite{achiam2023gpt4}, \texttt{Gemini}~\cite{team2023gemini}, \texttt{Claude}~\cite{anthropic2024claude}).), tasks \texttt{(CLRS30}~\cite{velivckovic2022clrs}, \texttt{NP-Hard-Eval}~\cite{fan2023nphardeval}, Custom Eval Suite) and seeds $\{0,1,2\}$. Our results suggest Route 3 (Code + Python Execution) is optimal (\cref{fig:main,fig:per_task_accuracy}). To understand where language is bottlenecked, we operationalize our framework below. 

When analyzing the pair Route 1 v.s. Route 2 (\cref{sec:rep_analysis}), we fix the execution phase to be a LLM with NL reasoning, and vary the representation \yu{representation modality} of its Chain-of-Thought to be either code or NL. Because our goal is to optimize the end-to-end reasoning pipeline rather than any particular instantiation of it, we evaluate representations and executors in terms of their computation-constrained optimal risk (hereafter optimal risk; \cref{def:risk}). This criterion captures the best achievable performance within a model family and abstracts away suboptimal prompt or executor choices. We propose sufficient conditions such that the optimal risk of code is never worse than language up to a small amount of error $\varepsilon$ (\cref{prop:dominance}). Furthermore, we empirically verify these conditions through two experiments that show that natural language is an approximate garbling of code (\cref{sec:dist_similarity,sec:func_similarity}).

We further analyze Route 2 v.s. Route 3 (\cref{sec:route2v3}), identifying execution to be the bottleneck. We find that when code is correct, execution always achieves lower expected loss than language, and when code is incorrect, the probability that Route 2 $>$ Route 3 is small. On an analysis of hardness scaling, we find that code execution (Route 3) scales much better as problems become more difficult, and that the probabilities of Route 2 $>$ Route 3 when code is wrong reduces as problems get harder. 
% \yu{too dense, each paragraph should only have one focus, and some claims sound handwavy/overconfident, e.g., hinting at trace generation not being the bottleneck, (i.e. looks similar),}



% that works suggest that the \textbf{solver route}, translating problems into solver-executable code representations and delegating execution, often outperforms the \textbf{direct route}, reasoning end-to-end in NL, on logic- and algorithmic-style complex reasoning tasks. However, for algorithmic reasoning alone, there is still no systematic analysis comparing the two routes, clarifying why LLMs perform better via the solver route versus the direct route. 
% \yu{TODO: add more recent citations.}

% Despite these trends, for algorithmic reasoning it remains underexplored whether LLMs perform better via the \textbf{direct route}, reasoning end-to-end in NL, or the \textbf{solver route}, translating problems into solver-executable representations and delegating execution (Todo: despite.. add reference for there are many empirical results showing code is better).}


% A principled direct comparison is challenging since the routes operate over different representation spaces, i.e., NL traces versus solver-executable programs, and rely on different execution mechanisms, which prevents step-by-step alignment.
% Specifically, sample-complexity comparisons~\cite{bai2023transformers} are ill-posed here because the two routes learn fundamentally different objects, so there is no common formal target and metric to compare. Computational-complexity arguments~\cite{merrill2023expressive} are also not a clean discriminator here because the two routes incur fundamentally different execution-dependent costs. We therefore compare the two routes via statistical difficulty, using optimal achievable end-task Bayes risk~\cite{xie2021explanation}.

% In this paper, we propose a three-route framework that makes this comparison tractable by introducing an additional intermediate \textbf{simulation route}, where the model performs the same code translation but simulates execution in NL, other than the \textbf{direct route} and \textbf{solver route}, and verbalizes the representation using Chain-of-Thought~\cite{wei2022chain} as shown in Fig.~\ref{fig:three_routes}. Using this framework, we characterize why algorithmic reasoning favors the solver route, and empirically show that solver-based pipelines are significantly better for a broad class of algorithmic tasks. This framework also enables for a tractable theoretical comparison of the routes.

% showing that the simulation route is at least as good as the direct route under mild regularity assumptions: the direct route is a Blackwell-garbled (information-degraded) version of the simulation route, and thus cannot be more informative for downstream decision-making. Finally, we empirically demonstrate that the solver route substantially outperforms the simulation route, highlighting additional gains from reliable external execution.

\begin{figure*}[t]
    \centering
    \vspace{-5pt}
    \includegraphics[width=1\textwidth]{images/CleanShot 2026-01-28 at 16.25.56@2x.png}
    \vspace{-25pt}
    \caption{\textbf{Prompt templates for three-route evaluation.} \textbf{Route~1 (NL)}: LLM reasons in natural language only, code forbidden. \textbf{Route~2 (Sim)}: LLM generates Python \texttt{solution()} then simulates execution in NL. \textbf{Route~3 (Code)}: Same code executed in Python runtime. This isolates execution mechanism while controlling translation.}
    \label{fig:route_prompts}
\end{figure*}

% Using our framework, we consistently observe a three-route ordering across algorithmic reasoning tasks: the solver route performs best, followed by the simulation route, and then the direct route (\cref{fig:main,fig:per_task_accuracy}). Moreover, the solver route's advantage widens as task difficulty increases, with 4.01$\times$ odds of getting a correct answer for code over natural language reasoning. We evaluate our framework on CLRS30~\cite{velivckovic2022clrs}, NP-Hard-Eval~\cite{fan2023nphardeval}, and a custom suite of algorithmic problems with controllable difficulty (addition, multiplication, LCS, rod cutting, knapsack, and ILP variants: assignment, production, and partition) across a broad range of models, spanning weaker open-source LLMs (e.g., Mistral~\cite{jiang2023mistral}, LLaMA~\cite{touvron2023llama}, Qwen~\cite{yang2024qwen2}) and stronger closed-source systems (e.g., OpenAI~\cite{achiam2023gpt4}, Gemini~\cite{team2023gemini}, Claude~\cite{anthropic2024claude}). We find that, averaged over tasks and models, the solver route achieves XX\% accuracy, outperforming the simulation route (XX\%) and the direct route (XX\%) with statistical significance (XX). 

% Theoretically, we formalize the three reasoning routes as statistical experiments in the sense of Blackwell~\cite{}. We show that, under mild regularity assumptions, natural language reasoning traces can be viewed as a garbling of code-based representations, inducing a Blackwell ordering in which code is at least as informative as natural language for downstream decision-making. This implies that code-based reasoning achieves weakly lower Bayes risk for algorithmic tasks. We further model execution as an information channel and show that delegating execution to an external solver corresponds to a deterministic channel, while LLM-based simulation introduces additional stochastic noise. Together, these results establish a principled ordering—natural language $\leq$ code simulation $<$ code execution—providing an information-theoretic explanation for the empirical advantages of solver-based pipelines.

%Introduce the setting and the problem. Make sure to narrow down the scope. 
% Consider the algorithmic task of computing arithmetic operations encoded in natural language. We wish to compute:
% \vspace{-5pt}
% \begin{equation*}
%     p(\texttt{Y=3} | \texttt{X=What is one plus two?})
% \end{equation*}
% \vspace{-5pt}
% The language model forward pass could decide (1) to generate the solution directly with natural language reasoning \cite{wei2022chain} (2) translate the problem into code and use a solver \cite{gao2023pal}. This paper provides empirical evidence that $\mathrm{Arm~1} < \mathrm{Arm~3}$ quantified by end-task accuracy. A body of work shows that this pipeline is generally effective \cite{lyu2023faithful, pan2023logic}, further evidenced by the empirical success of tool-use\footnote{Here we primarily refer to solver-based tool-use as opposed to knowledge-intensive tool-use like RAG}. However, little progress has been made on explaining \emph{why} solver-based tools lead to higher end-task accuracy than natural language reasoning.

%Why the problem is hard theoretically. 
% One might be tempted to prove a statistical advantage by showing that sample complexity to learn code is less than natural language because code is structured, but will quickly find that this problem becomes intractable due to the hardness of capturing natural language under a mathematical framework. The same problem arises when attempting to use approximation theory to provide evidence that DNNs can better learn compositional or structured languages (like code) than natural language. Broadly speaking, the problem is challenging because the inputs and outputs are different, one is a structured language, the other is an unstructured language. This drastically complicates comparison. Or, one might be tempted to show a computational advantage by showing that code unlocks a new level of expressivity \cite{merrill2023expressive}. However, one will find that whether using a solver or not will not overcome the hardness of the problem. E.g., we can prove by contradiction that using a solver will not allow us to better solve NP-Hard problems, unless P=NP.  

%What we did 
% As a solution, we leverage a Bayesian Inference paradigm to reason about the two different settings \cite{xie2021explanation}. Doing so, enables us to break down the algorithmic reasoning pipeline into two distinct phases (1) Translation $\in \{\mathrm{Code}, \mathrm{NL}\}$  (2) Execution $\in \{\mathrm{LLM~Reasoning}, \mathrm{Solver~Execution}\}$ \cite{lyu2023faithful, pan2023logic}. Enumerating the valid combinations, we obtain three pairs, Arm~1: $\{\mathrm{NL~Gen}, \mathrm{LLM~Reasoning}\}$, Arm~2: $\{\mathrm{Code~Gen}, \mathrm{LLM~Reasoning}\}$, Arm~3: $\{\mathrm{Code~Gen}, \mathrm{Solver~Execution}\}$. Introducing Arm~2 makes the problem tractable. 

%Why the problem hasn't been solved empirically. 
% Empirically, this framework enables controlled comparisons. A rigorous comparison has not been instantiated because it is hard to control the experiment and determine what representation is actually being used---whether code, natural language, or something else. We overcome this by verbalizing the representation using Chain-of-Thought. We provide statistical evidence for the alternative hypothesis $\mathrm{Arm~3} > \mathrm{Arm~2} > \mathrm{Arm~1}$ on (Gemma, Deepseek, Llama), we demonstrate that generating code and executing leads to 78\% accuracy on (ILP, DP, Arithmetic) tasks, over 30\% for code simulation and just over 20\% for natural language reasoning across 5 models. Our results are computed over 5 seeds, and a Friedman Chi-square gives results a test statistic of 9277.32 and 8369.34 for deepseek and llama, enabling us to reject the null in favor of the alternative. 

% Theoretically, we first compare $\mathrm{Arm~1} < \mathrm{Arm~2}$. Bayesian Inference shows us that the LLM implicitly does multi-class classification to the right algorithm. We utilize information theory to capture natural language and code under the same framework, forgoing using grammars or other mathematical frameworks that are intractable. We reduce the comparison of Bayes Error to that of comparing cross-entropy. A intermediate step using mutual information makes the proof interpretable: we prove that the mutual information between the CoT and the final answer is higher when conditioned on code representations over natural language representations. We variationally lower bound the mutual information using cross-entropy of a proposal distribution parameterized by a logistic regression. Since we only care about orderings, we subtract to overcome the intractability of estimating the differential entropy, reducing the comparison of mutual information to that of cross-entropy. The cross-entropy is measured empirically using logistic regression on TF-IDF features and Bert-base-uncased features, showing that code has lower cross-entropy than NL and achieves higher accuracy when classifying the correct algorithm. The difference is statistically significant (F-test, $p < 0.05$). %no need to say the details of reduction and not using LLM uncalibrated probs? 

% Then we compare $\mathrm{Arm~2} < \mathrm{Arm~3}$. This difference is easily explained using a communication channel model of LLM forward-pass. We show that in the case where Arm~2 $>$ Arm~3, i.e. when the code generated is not executable or wrong, yet the LLM reasoning obtains the correct answer, that this occurs rarely. In other words, generally $\mathrm{Arm~3} > \mathrm{Arm~2}$. 

% Piecing these results together, we show that $\mathrm{Arm~1} < \mathrm{Arm~2} < \mathrm{Arm~3}$, verifying the hypothesis.




% Understanding this problem is crucial as we move towards compositional AI systems rather than monolithic architectures.
                 
Our main contributions are:
\begin{enumerate}
    \item A \textbf{three-route framework} (\cref{sec:framework}) for tractable comparison between code and natural language representations via an intermediary (code generation with LLM execution).
    \item \textbf{Empirical validation} (\cref{sec:empirical}) demonstrating that our framework's ordering holds such that Route 1 $\leq$ Route 2 $<$ Route 3.
    \item A \textbf{systematic study} (\cref{sec:rep_analysis,sec:route2v3}) ruling out trace generation, and solidifying execution as the bottleneck in end-to-end algorithmic reasoning performance when using language.
\end{enumerate}


\begin{figure*}[t]
    \centering
    \vspace{-5pt}
    \includegraphics[width=1\textwidth]{images/combined_accuracy_delta.png}
    \vspace{-25pt}
    \caption{\textbf{Code + Solver Execution performs better than Direct NL reasoning quantified by end-task accuracy overall and within paired instances.} Paired instance contrasts are shown in the bootstrap distributions. Results are averaged over CLR30, NPHardEval, Custom Eval Suite, 3 seeds and 6 models. }
    \label{fig:main}
\end{figure*}
\section{Three-Route Framework}
\label{sec:framework}
Our central research question (RQ) is: \emph{Is code $>$ NL for algorithmic reasoning?} To begin to answer this question, we first introduce our three-route framework which enables tractable comparison by disentangling \emph{reasoning representation} (hereafter called Traces) and \emph{reasoning execution} (hereafter called Executors) and constructing an intermediary bridge (Route 2) for pairwise comparison. This section introduces the task (\cref{def:task}), notation (\cref{def:trace}), and definitions (\cref{def:risk}, \cref{def:route}).

% \yu{It's not only information representation, reasoning modality/formalization might be a better word, since it includes both representation and intermediate reasoning} from differences in \emph{execution}, enabling tractable and interpretable comparisons. \yu{We first present basic definitions in the paper through \S\ref{def:task} to \S\ref{def:risk} and then introduce the three-arms framework in \S\ref{def:arm}. Note: always try to conclude/foresee the section in the first paragraph} 

\subsection{Task and Loss}
\label{def:task}
% \yu{Let $p(x)$ be the test distribution over task instances, and let $X \sim p(x)$ denote a sampled instance (problem statement and inputs). Q: test distribution across all possible algorithmic tasks or just one?}
For a gold evaluation distribution over task instances $i$ specified by (algorithm, input variables, seed), let $X \sim p(x)$ denote a task instance (problem statement and inputs), and let $Y^*(X) \in \mathcal{Y}$ denote the ground-truth output that is unique, fixed, and externally verifiable. 
% \yu{, where 
% $\mathcal{Y}$ is the set of possible outputs. Q: is the output fixed/unique/verifiable, if so, add unique, fixed, and externally verifiable ground-truth output}
We evaluate performance under $0$--$1$ loss:
$$\ell(y,x) := \mathbf{1}\{y \neq Y^*(x)\}.$$
All risks are taken with respect to the evaluation distribution $p(x)$.

\subsection{Trace Generators and Executors}
\label{def:trace}
We model each reasoning pipeline as a two-stage stochastic process with \emph{(1) trace generation} and \emph{(2) execution}. 
\paragraph{Trace generation.} We define a \emph{trace} as an object that stores intermediate reasoning used to solve a corresponding task (e.g. NL Chain-of-Thought or a program). A \emph{trace generator} is a Markov kernel
\[
E : \mathcal{X} \to \Delta(\mathcal{Z}), \qquad
Z \sim p_E(z \mid x),
\]
% \yu{$p_E$ looks a bit wierd}
which produces an auxiliary trace $Z$ given the task instance.

 
\paragraph{Execution.} We define \emph{execution} as a procedure that consumes a trace, e.g. an LLM forward pass that takes as input a reasoning trace and outputs an answer, or executing a program in an external runtime. An \emph{executor} is a (possibly randomized) Markov kernel
\[
\rho : \mathcal{X} \times \mathcal{Z} \to \Delta(\mathcal{Y}),
\]
mapping the observed instance and trace to a final output.

The induced conditional output distribution is
\[
P_{\rho,E}(Y = y \mid X = x)
= \int \rho(y \mid x, z)\, p_E(z \mid x)\, dz .
\]

The population risk of a pipeline $(E,\rho)$ is
\begin{align*}
& R(E,\rho)
:= \mathbb{E}\bigl[\ell(\hat{Y}, X)\bigr],
\qquad \\
& Z \sim p_E(\cdot \mid X),\;
\hat{Y} \sim \rho(\cdot \mid X, Z).
\end{align*}

\subsection{Computation-Constrained Optimal Risk}
\label{def:risk}
To reflect inference-time computational constraints, we evaluate executors within restricted families.
For a trace space $\mathcal{Z}$ (e.g. code or NL), let
\[
\mathcal{H}_{\mathcal{Z}} \subseteq
\{ \rho : \mathcal{X} \times \mathcal{Z} \to \Delta(\mathcal{Y}) \}
\]
denote a class of executors realizable under a fixed model family, e.g. Mistral (see \cref{sec:cond1}).
% and bounded inference protocol\yu{which will be introduce in \S\ref{} add a reference here}.
We define the computation-constrained optimal risk of a trace generator $E$ as
\[
R^*_{\mathcal{H}}(E)
:= \inf_{\rho \in \mathcal{H}} R(E,\rho).
\]
This differs from classical Bayes risk, which optimizes over all measurable decision rules.



\subsection{The Three Routes}
\label{def:route}
We consider three reasoning pipelines (``routes''), illustrated in \cref{fig:three_routes}. Each route is represented as a pair $(E,\rho)$ consisting of a trace generator $E$ and an executor $\rho$, evaluated via the risk $R(E,\rho)$ (\cref{def:trace}) and the computation-constrained
optimal risk $R^*_{\mathcal H}(E)$ (\cref{def:risk}).
% \yu{A concrete example of the three routes can be referred to in Figure~\ref{fig:main}.}
% \yu{I will add some NL descriptions which can be referred to in other sections. }
\paragraph{Route 1 (Direct Natural Language).}
Route~1 represents standard NL reasoning: the model first produces a natural-language (Chain-of-thought) trace and then the same model is used as the LLM-based executor, conditioning on the trace to generate a final answer. Formally, a natural-language trace generator $E_{\mathrm{NL}}$ produces traces
$Z_{\mathrm{NL}} \sim p_{\mathrm{NL}}(\cdot \mid X)$,
paired with an executor family
\[
\mathcal{H}_{\mathrm{NL}}
\subseteq
\{ \rho : \mathcal{X} \times \mathcal{Z}_{\mathrm{NL}} \to \Delta(\mathcal{Y}) \}.
\]
\paragraph{Route 2 (Code + NL Simulation).}
Route~2 uses \emph{code} as the trace modality, but keeps execution ``in-model'': the LLM instead simulates the generated code in natural language, rather than executing it in an external environment. This route isolates the effect of using a code trace while holding the LLM-based executor class fixed. Formally, a code trace generator $E_{\mathrm{Code}}$ produces executable representations
$Z_{\mathrm{C}} \sim p_{\mathrm{Code}}(\cdot \mid X)$,
paired with an executor family
\[
\mathcal{H}_{\mathrm{C}}
\subseteq
\{ \rho : \mathcal{X} \times \mathcal{Z}_{\mathrm{C}} \to \Delta(\mathcal{Y}) \}
\]
corresponding to language-model-based simulation of code execution.
\paragraph{Route 3 (Code + Solver Execution).}
Route~3 uses the same code trace generator $E_{\mathrm{Code}}$ as Route~2, producing
$Z_{\mathrm{C}} \sim p_{\mathrm{Code}}(\cdot \mid X)$,
but pairs it with a deterministic executor corresponding to external code execution.
Let $\mathrm{Exec} : \mathcal{X} \times \mathcal{Z}_{\mathrm{C}} \to \mathcal{Y}$ denote the (fixed) external runtime
that executes the code trace $z$ on instance $x$ and returns an output.
Our executor is
\begin{align*}
& \rho_{\mathrm{Exec}}(y \mid x,z) := \mathbf{1}\{y = \mathrm{Exec}(x,z)\}.
\end{align*}
The corresponding executor family is the singleton $\mathcal{H}_{\mathrm{Exec}} := \{\rho_{\mathrm{Exec}}\}.$

% \subsection{Interpreting the Three routes}

% In all arms, the executor observes the full task instance $X$; the trace $Z$ provides auxiliary information. Accordingly, comparisons between arms are conditional on $X$ and should be interpreted as statements about what auxiliary representations can be simulated from others, and how execution mechanisms affect achievable risk.

% Performance comparisons are expressed in terms of computation-constrained optimal risks $R^*_{\mathcal{H}}(E)$, which isolate differences due to representation and execution under fixed inference constraints.

% \subsection{Roadmap}
% \label{roadmap}
% Section~\ref{sec:empirical} instantiates this framework with concrete models, prompts, datasets, and statistical tests. Section~\ref{sec:performance} provides empirical evidence motivating assumptions about the relationship between natural-language and code traces. Section~\ref{sec:theory} analyzes the three arms theoretically within this framework.

\begin{figure*}[t]
    \centering
    \vspace{-5pt}
    \includegraphics[width=1\textwidth]{images/main_combined.png}
    \vspace{-25pt}
    \caption{ \textbf{Code + Solver Execution scales better than natural language reasoning when problems get harder, both within-task and on average, interpolating and extrapolating.} $\tau$ is used to control digit length for arithmetic, table dimensionality for dynamic programming, and constraint matrix dimensionality for integer linear programming. The same data and setup is used from \cref{fig:three_routes}. }
    \label{fig:per_task_accuracy}
\end{figure*}
\section{Evaluating the Three-Route Framework}
\label{sec:empirical}
Our evaluation aims to answer the research questions (RQ): 
\begin{enumerate}
    \item[1)] Does $\text{Route~1} \simeq \text{Route~2} < \text{Route~3}$ hold? (\S\ref{sec:instantiation}, \S\ref{sec:performance})
\end{enumerate}
 % RQ 1 is central to the main thesis, and RQ 2 provides motivation to offload to an executor, as well as empirical foundations for the assumptions in  Section~\ref{sec:theory}.


\subsection{Experimental Instantiation of the Three Routes}
\label{sec:instantiation}

To draw  conclusions about the effect of modality in the trace generation or different execution methods, we ensure one stage is always fixed. For Route 1 vs Route 2, we fix the execution phase by using the same LLM reasoning model forward pass, but use different modalities in the trace generator. For Route 2 vs Route 3, we fix the trace generator which outputs code, then use different execution methods, either LLM reasoning model forward pass, or a python3 runtime. Below, let $X_i$ be the task instantiation of instance $i$. Structured JSON outputs are enforced for sampling in all routes
(\cref{fig:route_prompts}). 

\paragraph{Sampling Route 1.}  Route 1 prompts an LLM $E_{NL} $ to function as a trace generator $Z_{NL} := E_{NL}(X_i)$. The trace is then fed into the same LLM $\rho_{NL} = E_{NL}$ to produce an answer $Y_{i}^{(NL)} := \rho_{NL} (Z_{NL})$. The prompt instructs the model to never use code, and to output a structured rationale and answer (see \cref{fig:route_prompts}). Operationalized, one experimental run for Route 1 is encapsulated in a single LLM reasoning model's forward pass without pause. 
\paragraph{Sampling Route 2.} Similarly, Route 2 uses a LLM $E_{Code}$ as a trace generator for code modality $Z_{Code} := E_{Code}(X_i)$. The trace is then fed into the same LLM $\rho_{Sim} = E_{Code}$ to produce an answer $Y_{i}^{(Sim)} := \rho_{Sim} (Z_{Code})$. Operationalized, one experimental run for Route 2 is encapsulated in a single LLM reasoning model's forward pass without pause. Prompt templates are controlled to be as similar as possible, with
Route~2 differing only by the inclusion of a code-generation and
simulation instruction (\cref{fig:route_prompts}). We prompt the model to output a program snippet, structured reasoning over the code, followed by an answer. 
\paragraph{Sampling Route 3.} Route 3 takes the exact same code trace $Z_{Code}$ as Route 2, except it runs it through a python3 runtime $\rho_{Exec}$, and retrieves the solution $Y_i^{(Exec)} = \rho_{Exec}(Z_{Code})$. The python3 runtime has access to 5 standard scientific libraries: \texttt{Scipy, Numpy, Pandas, PuLP,} and \texttt{Pytorch}. 
\paragraph{Observed outcomes.}
For each problem instance $i$, we observe paired Bernoulli outcomes \[\bigl(Y_i^{(\mathrm{NL})},\; Y_i^{(\mathrm{Sim})},\; Y_i^{(\mathrm{Exec})}\bigr).\]
% Let $X_i$ denote the original problem instance and $C_i$ the generated
% code for instance $i$. The tuple $(X_i, C_i)$ is held fixed across both
% arms. Arm~2 produces
% $(X_i, C_i) \;\mapsto\; Y_i^{(\mathrm{Sim})}$
% via language-model-based simulation, while Arm~3 executes the same code
% using an external Python runtime,
% $(X_i, C_i) \;\mapsto\; Y_i^{(\mathrm{Exec})}.$
% For notational symmetry, $X_i$ is included in both mappings, although it
% is ignored by the external executor.
\paragraph{Data and models.}
We evaluate on the CLRS-30 benchmark ($n=500$), the NP-Hard-Eval
benchmark ($n=540$), and a custom fine-grained evaluation suite
($n=540$), across three random seeds $\{0,1,2 \}$. Tasks span Arithmetic, Dynamic
Programming, and Integer Linear Programming (ILP), with difficulty
controlled by a parameter $\tau$.
We evaluate both closed-source models (\texttt{Claude Haiku~4.5, GPT-4o-mini, Gemini~2.5 Pro})
and open-source models (\texttt{Mixtral-8x22b-Instruct, Codestral-22B}). 

% Models with more than
% 50\% JSON parse failures are excluded to avoid confounding instruction
% following with reasoning ability.

% \paragraph{Prompting and execution.}
% In Arm~1, models are instructed to reason without using code and to
% output a structured rationale and answer.
% In Arm~2, the same prompt is augmented with instructions to generate
% code and simulate its execution.
% In Arm~3, the generated function is executed directly in a Python~3
% runtime with access to standard scientific libraries.



\subsection{End-to-End Performance Comparison}
\label{sec:performance}

\paragraph{Pairwise statistical tests.}
We evaluate pairwise route differences using McNemar tests on paired
Bernoulli outcomes.
For Route~1 vs.\ Route~2, the null hypothesis is
\begin{align*}
& H_0:\;
\Pr(Y^{(\mathrm{NL})}=1,\,Y^{(\mathrm{Sim})}=0) \\
& =
\Pr(Y^{(\mathrm{NL})}=0,\,Y^{(\mathrm{Sim})}=1),
\end{align*}
with an analogous null for Route~2 vs.\ Route~3.

Effect sizes are reported as paired accuracy differences, e.g.,
\[
\Delta_{\mathrm{Sim-NL}}
=
\mathrm{Acc}(\mathrm{Sim}) - \mathrm{Acc}(\mathrm{NL}),
\]
with 95\% confidence intervals estimated via cluster bootstrap
resampling over instances.
Holm--Bonferroni correction is applied to control family-wise error rate of the above 2 marginal pairwise tests on fully pooled data
at $\alpha=5\%$. %what family wise? Across difficulties? Add tau analysis later. 

\paragraph{Difficulty scaling.}
To analyze how performance varies with task difficulty, we fit a generalized linear mixed-effects model (GLMM)
for binary accuracy $Y_i \in \{0,1\}$ with a logistic link:
\[
Y_i \mid u_{\text{inst}[i]}, u_{\text{seed}[i]} \sim \mathrm{Bernoulli}(p_i),
\]
\[
\mathrm{logit}(p_i)
=
\alpha
+
\beta_{\mathrm{route}_i}
+
\gamma\,\tau_i
+
\delta_{\mathrm{route}_i}\,\tau_i
+
u_{\text{inst}[i]}
+
u_{\text{seed}[i]},
\]
where $u_{\text{inst}[i]} \sim \mathcal{N}(0,\sigma^2_{\text{inst}})$ and
$u_{\text{seed}[i]} \sim \mathcal{N}(0,\sigma^2_{\text{seed}})$ are random intercepts.
Route and difficulty are modeled as fixed effects (including an route $\times$ difficulty interaction),
with instance and seed modeled as random effects.

\begin{figure*}[t]
    \centering
    \vspace{-5pt}
    \includegraphics[width=\textwidth]{images/CleanShot 2026-01-27 at 11.05.15@2x.png}
    \vspace{-25pt}
    \caption{\textbf{Translation and discrimination prompts.}
    \textbf{(a) Translator}: Converts code to NL reasoning step-by-step, mimicking native reasoning.
    \textbf{(b) Discriminator}: Judge models classify traces as ``Native NL'' or ``Translated''.
    Used in Section~\ref{sec:rep_analysis}.}
    \label{fig:translator_discriminator}
\end{figure*}

\paragraph{Results.}
Across all benchmarks, we reject the global null hypothesis
($p < 0.001$ after correction).
\emph{Post-hoc tests show a statistically significant advantage of Route~3
over Route~2}, with a positive paired accuracy gap excluding zero in the
95\% confidence interval.
For Route~1 vs.\ Route~2, although the null is rejected ($p = 0.04$), the
bootstrap confidence interval overlaps zero, indicating that we cannot
conclude a meaningful difference between natural-language reasoning
and code simulation under this evaluation.
Performance gaps widen with increasing task difficulty: Route~3 remains
stable while Route~1 degrades sharply, with code execution achieving
approximately $4.1\times$ higher odds of correctness.

% \subsection{Representation-Level Analysis: Does Language Add Information?}
% \label{sec:rep_analysis}

% The theoretical results in Section~\ref{sec:theory} rely on the premise
% that, for the evaluated tasks, models, and prompts, natural-language
% reasoning traces do not contain decision-relevant information beyond
% what is already present in code representations.
% We empirically evaluate this premise by analyzing both distributional
% and functional similarity between native NL reasoning and NL obtained
% by translating code.
\section{Route 1 vs Route 2 Analysis}
\label{sec:rep_analysis}
The key research question in this section is: \textbf{In the scenario when our NL reasoning and Code reasoning are optimal, do NL reasoning traces contain decision-relevant information beyond what is present in code representations? (\S\ref{sec:dist_similarity}, \S\ref{sec:func_similarity})} One might argue that, when optimizing a code-execution pipeline end to end, the trace generation stage could benefit from representational properties unique to natural language, and that improving NL trace generation could therefore outperform code-based traces. Our analysis rules out this possibility at the level of computation-constrained optimal performance. Specifically, we show that
\[
R^*_{\mathcal{H}_{\mathrm{C}}}(E_{\mathrm{Code}})
\;\le\;
R^*_{\mathcal{H}_{\mathrm{NL}}}(E_{\mathrm{NL}})
+
\varepsilon.
\]
implying that, up to a small approximation error \S\ref{sec:func_similarity}, code traces are sufficient to match the best achievable performance of NL traces. Consequently, any performance gap between code-based and NL-based pipelines cannot be attributed to limitations of trace generation, but must arise from downstream execution.


\paragraph{Condition 1. ($\varepsilon$-garbling)}
\label{cond:simulability}
We say that NL traces are $\varepsilon$-simulable from code traces (for the executor family $\mathcal{H}_{\mathrm{NL}}$)
if there exists a translation kernel $T:\mathcal{Z}_{\mathrm C}\to\Delta(\mathcal{Z}_{\mathrm{NL}})$ such that,
letting $E_{\mathrm{Tran}} := T\circ E_{\mathrm{Code}}$,
\begin{align*}
& \Delta_{\mathcal{H}_{\mathrm{NL}}}(E_{\mathrm{NL}},E_{\mathrm{Tran}})
:= \\
& \sup_{\rho\in\mathcal{H}_{\mathrm{NL}}}
\left|
R(E_{\mathrm{NL}},\rho)-R(E_{\mathrm{Tran}},\rho)
\right|
\le \varepsilon,
\end{align*}
and such that $\mathcal{H}_{\mathrm{C}}$ is closed under composition with $T$
(i.e., for each $\rho_{\mathrm{NL}}\in\mathcal{H}_{\mathrm{NL}}$,
$\rho_{T\circ \mathrm{NL}}\in\mathcal{H}_{\mathrm{C}}$).
In Section~\ref{sec:rep_analysis} and Section~\ref{sec:func_similarity} we estimate these quantities and find $\varepsilon$ is small. 

\begin{proposition}[Consequence of $\varepsilon$-garbling]
\label{prop:dominance}
If Condition~\ref{cond:simulability} holds, then
\[
R^*_{\mathcal{H}_{\mathrm{C}}}(E_{\mathrm{Code}})
\le
R^*_{\mathcal{H}_{\mathrm{NL}}}(E_{\mathrm{NL}})
+\varepsilon.
\]
\end{proposition}
\begin{proof}
Fix any $\delta>0$ and choose
$\rho^{\delta}_{\mathrm{NL}}\in\mathcal{H}_{\mathrm{NL}}$ such that
\[
R(E_{\mathrm{NL}},\rho^{\delta}_{\mathrm{NL}})
\le
R^*_{\mathcal{H}_{\mathrm{NL}}}(E_{\mathrm{NL}}) + \delta.
\]

By Condition~1, the composed executor that ``translates then decodes,''
\[
\rho^{\delta}_{T\circ \mathrm{NL}}(y \mid x, z_{\mathrm{C}})
:=
\int \rho^{\delta}_{\mathrm{NL}}(y \mid x, z_{\mathrm{NL}})
\,T(z_{\mathrm{NL}}\mid z_{\mathrm{C}})\,dz_{\mathrm{NL}},
\]
is realizable in the code-executor class, i.e.,
$\rho^{\delta}_{T\circ \mathrm{NL}} \in \mathcal{H}_{\mathrm{C}}$.

Now consider the following stochastic computation graph: first sample a code trace
$Z_{\mathrm{C}} \sim p_{\mathrm{Code}}(\cdot \mid X)$; then translate it to an NL trace
$Z_{\mathrm{NL}} \sim T(\cdot \mid Z_{\mathrm{C}})$; finally sample the answer
$\hat Y \sim \rho^{\delta}_{\mathrm{NL}}(\cdot \mid X, Z_{\mathrm{NL}})$.
This is exactly what $\rho^{\delta}_{T\circ \mathrm{NL}}$ implements: it performs the
translation step inside the executor and then applies the NL executor.

Equivalently, we may ``push'' the translation step into the trace generator.
By construction of the translated generator $E_{\mathrm{Tran}} := T \circ E_{\mathrm{Code}}$,
the marginal conditional distribution of $Z_{\mathrm{NL}}$ given $X$ is
\[
p_{\mathrm{Tran}}(z_{\mathrm{NL}} \mid x)
=
\int T(z_{\mathrm{NL}} \mid z_{\mathrm{C}})\,p_{\mathrm{Code}}(z_{\mathrm{C}} \mid x)\,dz_{\mathrm{C}}.
\]
Therefore, the joint distribution of $(X, Z_{\mathrm{NL}}, \hat Y)$ induced by
(i) $(E_{\mathrm{Code}}, \rho^{\delta}_{T\circ \mathrm{NL}})$ and
(ii) $(E_{\mathrm{Tran}}, \rho^{\delta}_{\mathrm{NL}})$ is the same, and hence they have
identical risk:
\[
R(E_{\mathrm{Code}},\rho^{\delta}_{T\circ \mathrm{NL}})
=
R(E_{\mathrm{Tran}},\rho^{\delta}_{\mathrm{NL}}).
\]


By Condition 1,
\[
R(E_{\mathrm{Tran}},\rho^{\delta}_{\mathrm{NL}})
\le
R(E_{\mathrm{NL}},\rho^{\delta}_{\mathrm{NL}}) + \varepsilon
\le
R^*_{\mathcal{H}_{\mathrm{NL}}}(E_{\mathrm{NL}}) + \delta + \varepsilon.
\]
Taking the infimum over $\mathcal{H}_{\mathrm{C}}$ and letting
$\delta\to 0$ completes the proof.
\end{proof}

\subsection{Validating Condition 1.}
\label{sec:cond1}
Condition 1 requires natural language to be an approximate garbling of code. To verify this, we ask the sub-questions as a proxy: 
\begin{enumerate}
    \item[1)] Does there exist a translator LLM such that the NL traces translated from code \emph{look} like the original NL traces (existence of $\varepsilon$-Translator)?  (\S\ref{sec:cond1})
    \item[2)] Does there exist a translator LLM such that the NL traces translated from code \emph{function} like the original NL traces quantified by end-task accuracy (closure under composition)? (\S\ref{sec:func_similarity})
\end{enumerate}
In essence, these questions ask whether translation during inference really can be done in either trace generation or execution interchangeably (equal after marginalizing over code traces), similar to the ideas presented in the proof. We find that a translator LLM does exist (e.g. \texttt{GPT-4o} as an example), such that the NL traces translated from code look and function like the original NL traces. This provides support for Condition 1 and suggests that NL does not provide decision-relevant information beyond code in the optimal risk regime. 

\subsubsection{Existence of $\varepsilon$-Translator}
\label{sec:dist_similarity}
We find that the distribution of reasoning traces produced in Route~1 can be approximated by post-processing code traces with a fixed transformation (\texttt{GPT-4o} with prompting).
Concretely, for each task instance $x$ and corresponding code $c$, we form two trace conditions:
\begin{align*}
    & (x, z_{\mathrm{NL}}), \quad z_{\mathrm{NL}} \sim p_{\mathrm{NL}}(\cdot \mid x), \quad \text{(\emph{Native})},\\
    & (x, \hat z_{\mathrm{NL}}), \quad \hat z_{\mathrm{NL}} \sim p_{\mathrm{Tr}}(\cdot \mid c), \quad \text{(\emph{Translated})},
\end{align*}
where $p_{\mathrm{Tr}}(\cdot\mid c)$ is induced by a fixed translator $T$ applied to code, i.e.,
$\hat z_{\mathrm{NL}} = T(c)$ with fixed prompting/in-context examples (Figure~\ref{fig:translator_discriminator}a).

\paragraph{Discrimination task.}
We formulate a binary classification task: a powerful zero-shot judge model is given $(x, z)$ and must predict whether $z$ is \emph{Native} or \emph{Translated} (Figure~\ref{fig:translator_discriminator}b).
This corresponds to distinguishing the joint distributions
\[
p(x)p_{\mathrm{NL}}(z\mid x)
\quad\text{vs.}\quad
p(x)p_{\mathrm{Tr}}(z\mid x).
\]
We use multiple judge models (e.g., Claude Opus 4.0, Gemini 2.5 Pro, Grok 4.1 Fast) to reduce dependence on a single discriminator.

\paragraph{Sampling and controls.}
We sample problem instances $x$ from a pool of previously collected CoT examples spanning different models\footnote{In our experiments, when we conditioned on the models for the source inputs, we find that they become more distinguishable. This suggests that between models, there are stylistic artifacts that are significant enough to differentiate them. Since this is orthogonal to the native vs translated signal, we did not condition on models. } , seeds, tasks, and difficulties, and construct paired examples with balanced labels.
Tasks used for discrimination are held out from the translator’s in-context examples to avoid trivial leakage.
As a calibration control, we also ask the same judges to discriminate \emph{raw code} vs.\ \emph{native NL} traces; high control accuracy indicates the judges are not underpowered.

% \paragraph{Metrics.}
% We report judge accuracy with Wilson confidence intervals, both overall and stratified by task/model when informative.

% \begin{figure*}[t]
%     \centering
%     \vspace{-5pt}
%     \includegraphics[width=0.68\textwidth]{images/discrimination_by_task.png}
%     \vspace{-5pt}
%     \caption{\textbf{Discrimination results.}
%     Judge models classify traces as Native vs.\ Translated.
%     Accuracy near chance indicates translated NL is distributionally similar to native NL.
%     The control task (code vs.\ NL) remains high, indicating calibrated judges.}
%     \label{fig:discrimination_by_task}
% \end{figure*}



\begin{figure}[t]
    \centering
    \includegraphics[width=\linewidth]{images/judge_discrimination_barplot.png}
    \vspace{-15pt}
    \caption{\textbf{Translated NL from code is indistinguishable to original NL traces, supporting the hypothesis that NL contains no more decision-related information than code.} Judges run zero-shot and are prompted to discriminate between native and translated, the translator is GPT-4o which gets 10-shot examples. Original data sources are pooled data from \cref{fig:route_prompts}. }
    \label{fig:judge_barplot}
\end{figure}

\begin{figure}[t]
    \centering
    \includegraphics[width=\linewidth]{images/native_vs_translated_scatter.png}
    \vspace{-25pt}
    \caption{\textbf{Translated NL and Native NL form local clusters in the embedding space and have high cosine similarity, explaining distributional indistinguishability.} We randomly sample and embed 200 native NL reasoning traces and 200 code-to-NL translations (produced by GPT-4o) using OpenAI's text-embedding-3-large and project them via t-SNE. Cosine similarities between Native-Native and Translated-Translated sample pairs from the same task.}
    \label{fig:translation_additivity}
\end{figure}

\begin{figure}[t]
    \centering
    \vspace{-5pt}
    \includegraphics[width=\linewidth]{images/translation_additivity.png}
    \vspace{-25pt}
    \caption{\textbf{Translated NL has the same functionality as the original native NL quantified by end-task accuracy. }. When concatenating the reasoning traces to the prompt, we get much higher accuracy as opposed to the baseline with just the prompt, indicating the model is using the reasoning. There are statistically significant differences between the baseline and concatenated prompts, but no statistically significant difference between the Translated NL and Native NL concatenations. Source data is randomly sampled from existing data used to generate \cref{fig:three_routes}, we use 1000 samples per model. }
    \label{fig:translation_additivity}
\end{figure}

\paragraph{Results.}
Across $\sim$2000 samples per judge ($\sim$6000 total across three judges), discrimination accuracy is near chance (50\%), with 50\% included in the Wilson CI. The control discrimination between raw code and native NL is substantially higher (e.g., $\approx$79\%), indicating the judges can detect obvious representation differences but \emph{fail to distinguish translated vs.\ native NL}.

% (49.4\% with 95\% Wilson CI [47.2\%, 51.5\%]; Figure~\ref{fig:discrimination_by_task}).

% This suggests that, under our translation procedure, code-derived NL traces are distributionally similar to native NL CoT for most tasks.
% We observe a small number of task-level exceptions (e.g., rod-cutting and Kruskal-style MST) where discrimination may exceed chance, which we treat as potential failure modes of a single fixed translator or as task-specific stylistic artifacts. 

\subsubsection{Closure under Translation}
\label{sec:func_similarity}
We wish to show that whatever NL-executor can do after seeing an NL trace, the NL-executor can do after seeing a code trace, by first translating the code trace into an NL trace then behaving like the original NL-executor. We therefore test functional similarity using a downstream accuracy intervention.


\paragraph{Intervention setup.}
For a task instance $x$, we prompt a target language model in three conditions:
(1) Baseline:  x,
(2) Native: $ x \,\|\, z_{\mathrm{NL}},$
(3) Translated:  x $\,\|\, \hat z_{\mathrm{NL}},$ where $z_{\mathrm{NL}} \sim p_{\mathrm{NL}}(\cdot\mid x)$ is the native Route~1 trace and
$\hat z_{\mathrm{NL}}$ is obtained by translating the corresponding code to NL using the same translator procedure as in \S\ref{sec:dist_similarity}.
If translated NL loses decision-relevant information relative to native NL, we would expect
\[
\mathrm{Acc}(x \,\|\, \hat z_{\mathrm{NL}}) < \mathrm{Acc}(x \,\|\, z_{\mathrm{NL}}).
\]

\paragraph{Protocol and models.}
We run this test on 1000 held-out instances across multiple translator models (\texttt{Claude Haiku 4.5, Gemini-2.5-Flash, Mixtral-8x22B-Instruct}).
Crucially, in this experiment the translator model is matched to the original code generator (i.e., we translate code produced by the same model family), to avoid confounding functional losses with cross-model stylistic mismatch.

\paragraph{Results.}
We fail to reject the null hypothesis of equal performance, and observe overlapping 95\% confidence intervals between the Native and Translated conditions (Figure~\ref{fig:translation_additivity}).
This suggests that, under our evaluated settings, translating code into NL does not destroy decision-relevant information for downstream answering, consistent with the claim that NL CoT does not systematically add algorithmic advantage beyond what is in code. Qualitatively, we notice that the translated results are quite similar to the originals, though this seems heavily reliant on the translating prompt. 


\paragraph{Interpretation.} Interpretation. Our empirical results indicate that Condition 1 holds in the settings we study, implying that the best achievable performance using code traces matches that of natural-language traces up to a small error $\varepsilon$. In particular, we find no evidence that natural-language reasoning introduces novel algorithmic strategies beyond those already captured by code representations on algorithmic tasks. As a result, replacing NL traces with code traces in the generation stage does not constitute a performance bottleneck when optimizing the end-to-end reasoning pipeline. Instead, the remaining limitations are best attributed to the execution mechanism rather than the trace representation.

% The end-to-end results in Section~\ref{sec:performance} are consistent with the hypothesis. If true, the hypothesis would imply little benefits of NL over code for the trace generation stage. Thus, using code in the trace generation stage, and delegating to an external runtime in the execution stage which also avoids simulation noise would be the optimal choice (Arm 3). We test the hypothesis by answering two research questions.


% This section finds that NL does not contain decision-relevant information beyond what is present in NL, since translated NL looks both distributionally similar and functionally similar. Our statistical interpretation concludes that code is always as good as language for trace generation up to a $\varepsilon$ bound, which we empirically demonstrate is small via the functional similarity experiment. This suggests trace generation is not the bottleneck for language reasoning, pointing towards execution as the bottleneck (\cref{theory: route2vroute3}).
% The answers to these questions are themselves interesting, but also empirically motivate assumptions used in Section~\ref{sec:theory}.
\section{Statistical and Information Theoretic Foundations of Algorithmic Reasoning}
\label{sec:theory}
% Following \cite{blackwell1953equivalent}, we prove that $Arm 1 \simeq Arm 2 < Arm 3$ leveraging a similar breakdown framework as our experiments. We utilize information theory and statistical decision theory to sidestep representing the ambiguity of natural language and differences between NL and Code in a mathematical framework. \yu{Using the intuitions we gain in our hypothesis experiments in section~\ref{sec:exp_similarity}}, we prove that Arm 2 is at least as good as Arm 1, and that Arm 3 is always better than Arm 2. 

Following \citet{blackwell1953equivalent}, we formalize the comparison between our arms by separating (i) the \emph{information structure} available to the agent from (ii) the \emph{decision rule} used to map that information to an answer. Concretely, each arm induces an \emph{experiment} (a channel from task instances to traces) together with a restricted class of \emph{executors} (decision rules) that operate on the trace. This separation clarifies which comparisons are information-theoretic (Blackwell-style) versus computational (executor-class–dependent).

\subsection{Preliminaries}
\paragraph{State, experiment, and decision rule.}
Let $X \sim p(x)$ denote a task instance (problem statement plus inputs) drawn from a test distribution $p$. Each instance has a ground-truth output $Y^*(X) \in \mathcal{Y}$. We evaluate 0--1 loss
\[
\ell(y, x) := \mathbf{1}\{y \neq Y^*(x)\} \in \{0,1\}.
\]
An \emph{experiment} (information structure) is a conditional distribution over traces $Z \in \mathcal{Z}$ given the instance,
\[
E : x \mapsto p_E(z \mid x).
\]
A (possibly randomized) \emph{decision rule} or \emph{executor} is a conditional distribution over outputs given the instance and trace,
\[
\rho : (x,z) \mapsto \rho(y \mid x, z).
\]
Given $(E,\rho)$, the induced output distribution is
\[
P_{\rho,E}(Y=y \mid X=x) \;=\; \int \rho(y \mid x, z)\, p_E(z \mid x)\, dz.
\]

\paragraph{Arm-specific experiments and executors.}
Arm~1 (NL) corresponds to the experiment $E_{\mathrm{NL}}$ with trace $Z_{\mathrm{NL}}$ generated by $p_{\mathrm{NL}}(z_{\mathrm{NL}} \mid x)$ and executed by an NL executor $\rho_{\mathrm{NL}}(y \mid x, z_{\mathrm{NL}})$.
Arm~2 (Code) corresponds to the experiment $E_{\mathrm{Code}}$ with trace $Z_{\mathrm{C}}$ generated by $p_{\mathrm{Code}}(z_{\mathrm{C}} \mid x)$ and executed by a simulator executor $\rho_{\mathrm{Sim}}(y \mid x, z_{\mathrm{C}})$.

Our empirical results in Section~\ref{sec:exp_similarity} suggest that NL traces can be approximately obtained by post-processing (garbling) code traces. In Blackwell’s framework, a \emph{garbling} is a state-independent Markov kernel applied to the signal. Concretely, we posit that there exists a (possibly stochastic) translator
\[
T : \mathcal{Z}_{\mathrm{C}} \to \Delta(\mathcal{Z}_{\mathrm{NL}}), 
\qquad z_{\mathrm{NL}} \sim T(\cdot \mid z_{\mathrm{C}}),
\]
such that the \emph{translated} NL experiment induced by $E_{\mathrm{Code}}$ satisfies
\[
p_{\mathrm{translated}}(z_{\mathrm{NL}} \mid x)
:= \int T(z_{\mathrm{NL}} \mid z_{\mathrm{C}})\, p_{\mathrm{Code}}(z_{\mathrm{C}} \mid x)\, dz_{\mathrm{C}}.
\]
Under Assumptions~\ref{assump1}--\ref{assump2}, this translated NL experiment is close (in average conditional TV distance) to the native NL experiment $E_{\mathrm{NL}}$.

\paragraph{Composing translation with execution.}
Given a translator $T(z_{\mathrm{NL}} \mid z_{\mathrm{C}})$ and any NL executor $\rho_{\mathrm{NL}}(y \mid x, z_{\mathrm{NL}})$, define the induced code-side executor that (i) samples a translated NL trace and (ii) applies the same NL executor:
\[
\rho_{\mathrm{Sim}}(y \mid x, z_{\mathrm{C}})
:= \int \rho_{\mathrm{NL}}(y \mid x, z_{\mathrm{NL}})\, T(z_{\mathrm{NL}} \mid z_{\mathrm{C}})\, dz_{\mathrm{NL}}.
\]
This definition is purely a change of decision rule: the underlying experiment remains $E_{\mathrm{Code}}$.


% \begin{align*}
%     & \mathrm{P}(Y \mid X = x) := \int T(z \mid z_{\mathrm{Code}}) \, p_{\mathrm{Code}}(z_{\mathrm{Code}} \mid x) \, \mathrm{d}z_{\mathrm{Code}}
% \end{align*}
% In Arm~1 and Arm~2, each arm corresponds to an intermediate representation $Z$ (e.g. CoT) produced by channel $p(z \mid x)$ (e.g. LLM), and then choosing an output $Y$ via a randomized decision rule $\delta(y \mid x, z)$ (i.e. LLM test-time reasoning). 
\paragraph{Blackwell idealization and Le Cam approximation.}
In Blackwell’s framework, an experiment $E_2$ is (weakly) more informative than $E_1$ if there exists a state-independent Markov kernel $T$ such that $E_1$ can be obtained by garbling the signal from $E_2$. This \emph{exact} simulability implies that for every bounded-loss decision problem, the Bayes risk achievable under $E_2$ is no worse than under $E_1$ \citep{blackwell1953equivalent}.

In our setting, we do not assume exact equality of trace distributions. Instead, we assume that the native NL experiment is \emph{approximately simulable} from the code experiment via a state-independent kernel, and we quantify this approximation using total variation. This corresponds to a Le Cam–style comparison of experiments: approximate simulation implies an additive $\varepsilon$ bound on achievable risk for bounded losses \citep{lecam1986asymptotic}.

Finally, because our executors are computationally constrained (LLM-style inference), we evaluate these bounds over a restricted executor class $\mathcal{H}$ rather than over all measurable decision rules.
For an experiment $E$ with trace $Z \sim p_E(\cdot \mid X)$, define
\[
R^*_{\mathcal{H}}(E) \;:=\; \inf_{\rho \in \mathcal{H}} \; \mathbb{E}[ \mathbf{1}\{\hat Y \neq Y^*(X)\} ].
,
\qquad \hat Y \sim \rho(\cdot \mid X, Z).
\]
Here $\mathcal{H}$ denotes a \emph{restricted executor class} (a subset of Markov kernels from $(x,z)$ to $\mathcal{Y}$) capturing the computational mechanism available at inference time. In our setting, $\mathcal{H}$ is the family of executors implementable by the same model family under the fixed prompting and decoding protocol used in our experiments (i.e., LLM-based execution/simulation). We emphasize that $R^*(E)$ is \emph{not} the classical Bayes risk (which would optimize over all measurable decision rules), but a computation-constrained risk appropriate for comparing LLM-style executors.

When $E=E_{\mathrm{NL}}$ we write $R^*(E_{\mathrm{NL}})$, and when $E=E_{\mathrm{Code}}$ we write $R^*(E_{\mathrm{Code}})$. Our goal is to show that
\[
R^*(E_{\mathrm{Code}}) \;\le\; R^*(E_{\mathrm{NL}}) + \varepsilon.
\]


\textbf{Assumption 1 (State-independent garbling kernel).}
\label{assump1}
There exists a state-independent Markov kernel (translator) $T(z_{\mathrm{NL}} \mid z_{\mathrm{C}})$ such that the translated experiment
\[
p_{\mathrm{translated}}(z_{\mathrm{NL}} \mid x)
:= \int T(z_{\mathrm{NL}} \mid z_{\mathrm{C}})\, p_{\mathrm{Code}}(z_{\mathrm{C}} \mid x)\, dz_{\mathrm{C}}
\]
is well-defined for all $x$. This assumption only asserts the existence of a state-independent post-processing map; it does not require any notion of optimality for translation.


\textbf{Assumption 2.}
\label{assump2}
We assume that the original NL reasoning chain of thought is close to the translated NL on average. Let $p_{\mathrm{NL}}$ be the Arm~1 channel and $p_{\mathrm{translated}}(\cdot \mid x)$ be the translated NL channel. Assume an average conditional TV bound:
\[
\mathbb{E}_{X \sim p} \bigl[
d_{\mathrm{TV}}\bigl(p_{\mathrm{NL}}(\cdot \mid X), \, p_{\mathrm{translated}}(\cdot \mid X)\bigr)
\bigr] \leq \varepsilon,
\]
where
\[
d_{\mathrm{TV}}(P, Q) = \sup_{B} |P(B) - Q(B)|.
\]
In other words, averaged over task instances, the NL trace produced by Arm~1 is close in distribution to the NL traces obtained by translating the code trace (Arm~2) using the translator $T$. Section~\ref{sec:exp_similarity} provides empirical evidence for this assumption using distributional similarity proxies between native and translated traces.  We only assume approximate simulation of $E_{\mathrm{NL}}$ from $E_{\mathrm{Code}}$; we do not assume (or attempt to show) a reverse simulation of code traces from NL traces. Accordingly, our comparison is one-sided and does not establish Blackwell equivalence (nor zero deficiency in both directions).

\begin{proof}
We apply Blackwell’s simulation idea at the level of experiments: code traces can be post-processed into translated NL traces via the state-independent kernel $T$. We then relate performance under the native NL experiment to performance under the translated experiment using a total-variation continuity bound.

\textbf{Step 1 (Simulation via composed executor).}
Fix any NL executor $\rho_{\mathrm{NL}}(y \mid x, z_{\mathrm{NL}})$. Define $\rho_{\mathrm{Sim}}$ from $\rho_{\mathrm{NL}}$ and $T$ as
\[
\rho_{\mathrm{Sim}}(y \mid x, z_{\mathrm{C}})
:= \int \rho_{\mathrm{NL}}(y \mid x, z_{\mathrm{NL}})\, T(z_{\mathrm{NL}} \mid z_{\mathrm{C}})\, dz_{\mathrm{NL}}.
\]
Then the induced conditional output distribution under $(E_{\mathrm{Code}}, \rho_{\mathrm{Sim}})$ equals the conditional output distribution obtained by first sampling $Z_{\mathrm{C}} \sim p_{\mathrm{Code}}(\cdot \mid x)$, then sampling $Z_{\mathrm{NL}} \sim T(\cdot \mid Z_{\mathrm{C}})$, and finally applying $\rho_{\mathrm{NL}}$:
\[
P_{\rho_{\mathrm{Sim}}, E_{\mathrm{Code}}}(Y=y \mid X=x)
= \int \rho_{\mathrm{NL}}(y \mid x, z_{\mathrm{NL}})\, p_{\mathrm{translated}}(z_{\mathrm{NL}} \mid x)\, dz_{\mathrm{NL}}.
\]
In particular, for any bounded loss, the expected loss under $(E_{\mathrm{Code}},\rho_{\mathrm{Sim}})$ equals the expected loss under the translated experiment paired with $\rho_{\mathrm{NL}}$.

Thus,
\[
\mathbb{E}[\ell(Y_{\mathrm{Sim}}, X)] = \mathbb{E}[\ell(\hat{Y}_{\mathrm{translated}}, X)].
\]
\textbf{Step 2: Substitute translated NL and original NL via TV lemma.} Here we show that if two signals look similar, they perform similarly. Even if translated NL is not exactly native NL, bounded loss decision problems cannot exploit small distributional differences, a property of the continuity property of Bayes Risk. 

\begin{lemma}[Total variation risk continuity]
\label{lem:tv_continuity}
Let $X \sim p(x)$. Let $Z \mid X=x \sim P_x$ and $Z' \mid X=x \sim Q_x$. For any measurable $g:\mathcal{X}\times\mathcal{Z}\to[0,1]$,
\[
\Bigl|\mathbb{E}[g(X,Z)] - \mathbb{E}[g(X,Z')]\Bigr|
\;\le\;
\mathbb{E}_X\!\left[d_{\mathrm{TV}}(P_X,Q_X)\right].
\]
\end{lemma}
We use the convention $d_{\mathrm{TV}}(P,Q):=\sup_{B}|P(B)-Q(B)|$; under this convention, the bound above follows from the dual characterization of total variation (e.g., \citet[Ch. 2, Definition 2.4]{Tsybakov2009}).


For each $x$ and trace $z$, define $g(x, z) := \mathbb{E}_{y|x,z}[\ell(y, x)]$. Then $g(x, z) \in [0, 1]$. Note that
\begin{align*}
\mathbb{E}_{(Y_{\mathrm{NL}}, X)}[\ell(Y_{\mathrm{NL}}, X)] &= \mathbb{E}_{(X, Z_{\mathrm{NL}})}[g(X, Z_{\mathrm{NL}})], \\
\mathbb{E}_{(\hat Y_{\mathrm{Tr}}, X)}[\ell(\hat{Y}_{\mathrm{translated}}, X)] &= \mathbb{E}_{(X, \hat{Z}_{\mathrm{NL}})}[g(X, \hat{Z}_{\mathrm{NL}})].
\end{align*}

Applying the TV lemma with $P_x = p_{\mathrm{NL}}(\cdot \mid x)$ and $Q_x = p_{\mathrm{translated}}(\cdot \mid x)$:
\begin{align*}
&\bigl| \mathbb{E}[\ell(Y_{\mathrm{NL}}, X)] - \mathbb{E}[\ell(\hat{Y}_{\mathrm{translated}}, X)] \bigr| \\
&\quad = \bigl| \mathbb{E}[g(X, Z_{\mathrm{NL}})] - \mathbb{E}[g(X, \hat{Z}_{\mathrm{NL}})] \bigr| \\
&\quad \leq \mathbb{E}_X \bigl[ d_{\mathrm{TV}}(p_{\mathrm{NL}}(\cdot \mid X), \, p_{\mathrm{translated}}(\cdot \mid X)) \bigr]
\leq \varepsilon.
\end{align*}

Therefore, rearranging gives
\begin{align*}
&\mathbb{E}[\ell(\hat{Y}_{\mathrm{translated}}, X)] \leq \mathbb{E}[\ell(Y_{\mathrm{NL}}, X)] + \varepsilon \\
&\mathbb{E}[\ell(Y_{\mathrm{Sim}}, X)]
= \mathbb{E}[\ell(\hat{Y}_{\mathrm{translated}}, X)]
\leq \mathbb{E}[\ell(Y_{\mathrm{NL}}, X)] + \varepsilon.
\end{align*}
Since the construction above holds for an \emph{arbitrary} NL executor $\rho_{\mathrm{NL}} \in \mathcal{H}$, it also holds in particular for an NL executor that (approximately) minimizes risk under $E_{\mathrm{NL}}$. Let $\rho_{\mathrm{NL}}^* \in \arg\min_{\rho \in \mathcal{H}} \mathbb{E}[\ell(\hat Y, X)]$ under $E_{\mathrm{NL}}$, and let $\rho_{\mathrm{Sim}}^*$ be the induced executor on code traces obtained by composing $\rho_{\mathrm{NL}}^*$ with $T$:
\[
\rho_{\mathrm{Sim}}^*(y \mid x, z_{\mathrm{C}})
:= \int \rho_{\mathrm{NL}}^*(y \mid x, z_{\mathrm{NL}})\, T(z_{\mathrm{NL}} \mid z_{\mathrm{C}})\, dz_{\mathrm{NL}}.
\]
By Step~1 and Step~2,
\[
\mathbb{E}\bigl[\ell(\hat Y_{\mathrm{Sim}}^{(\rho_{\mathrm{Sim}}^*)}, X)\bigr]
\le
\mathbb{E}\bigl[\ell(\hat Y_{\mathrm{NL}}^{(\rho_{\mathrm{NL}}^*)}, X)\bigr] + \varepsilon.
\]
Finally, since $R^*(E_{\mathrm{Code}})$ is the infimum over \emph{all} executors in $\mathcal{H}$ applied to $E_{\mathrm{Code}}$, we have
\[
R^*(E_{\mathrm{Code}})
\le
\mathbb{E}\bigl[\ell(\hat Y_{\mathrm{Sim}}^{(\rho_{\mathrm{Sim}}^*)}, X)\bigr]
\le
R^*(E_{\mathrm{NL}}) + \varepsilon.
\]
\begin{framed}
\qquad \qquad \quad $R^*(E_{\mathrm{Code}}) \le R^*(E_{\mathrm{NL}}) + \varepsilon$
\end{framed}

\end{proof}
This proof follows the standard Le Cam/Blackwell logic: simulation of one experiment from another (exactly in Blackwell, approximately in Le Cam) yields risk bounds for bounded losses.

% \vspace{-20pt}
\subsection{Arm~2 $<$ Arm~3}
\textbf{Setup.}
Both Arm~2 and Arm~3 share the same underlying experiment $E_{\mathrm{Code}}$ that generates code traces $Z_{\mathrm{C}} \sim p_{\mathrm{Code}}(\cdot \mid X)$. They differ only in the executor applied to the same trace. Let $\rho_{\mathrm{Sim}} \in \mathcal{H}$ denote the LLM simulation executor, and let $\rho_{\mathrm{Exec}}$ denote deterministic execution by a Python interpreter $g$:
\[
\rho_{\mathrm{Exec}}(y \mid x, z_{\mathrm{C}}) := \mathbf{1}\{y = g(x, z_{\mathrm{C}})\}.
\]
We compare the expected 0--1 loss of these two executors under the shared experiment $E_{\mathrm{Code}}$.
\begin{proof}
Define the event
\[
C := \{ g(X, Z_{\mathrm{C}}) = Y^*(X) \},
\]
corresponding to semantically correct code.
Under $\ell(y,x)=\mathbf{1}\{y\neq Y^*(x)\}$, on the event $C$ the deterministic executor is correct with probability one:
\begin{align*}
& \mathbb{E}\!\left[\ell(\hat Y^{(\rho_{\mathrm{Exec}})},X)\mid C\right]=0 \\
& \mathbb{E}\!\left[\ell(\hat Y^{(\rho_{\mathrm{Sim}})},X)\mid C\right]\ge 0,
\end{align*}

Therefore, conditional on semantically correct code, deterministic execution weakly dominates LLM simulation in terms of conditional expected 0--1 loss under the shared experiment $E_{\mathrm{Code}}$.

\begin{framed}
\qquad \qquad \quad
$\mathbb{E}[\ell(\hat Y^{(\rho_{\mathrm{Exec}})},X)\mid C]
\;\le\;
\mathbb{E}[\ell(\hat Y^{(\rho_{\mathrm{Sim}})},X)\mid C]$
\end{framed}

\end{proof}

The only scenario where Arm~2 could outperform Arm~3 is when the generated code is \emph{incorrect}, yet the LLM ``recovers'' by reasoning to the correct answer despite the flawed code. We empirically quantify this recovery rate below. 

\textbf{Recovery reduces as tasks get harder.}
To assess the practical impact of recovery, we empirically measure the frequency of such events as task difficulty increases:
\begin{enumerate}[leftmargin=*]
    \item Arm~3 produces an incorrect answer (implying incorrect code generation), and
    \item Arm~2 produces the correct answer (implying successful LLM recovery).
\end{enumerate}

\cref{fig:recovery_final} presents the recovery analysis across all tasks and models. The recovery rate remains consistently low (typically $< 5\%$), indicating that LLM simulation rarely compensates for code generation errors. This confirms that Arm~3's advantage stems from reliable solver execution rather than Arm~2's inability to reason about code.

\begin{figure}[t]
    \centering
    \vspace{-5pt}
    \includegraphics[width=0.85\linewidth]{images/recovery_final.png}
    \vspace{-5pt}
    \caption{\textbf{Recovery rate.} How often Arm~2 succeeds when Arm~3 fails. Recovery rates are $<$5\% across all tasks, confirming Arm~3's advantage comes from reliable execution, not LLM compensation.}
    \label{fig:recovery_final}
\end{figure}



\section{Related Work and Discussion}

\textbf{Neuro-symbolic Learning.} This paper builds on research in neuro-symbolic integration \cite{graves_neural_2014, velickovic_neural_2021, reed_neural_2016, graves_hybrid_2016}, which combines neural networks with symbolic reasoning systems. These approaches are motivated by cognitive science \cite{schneider_controlled_2003, risko_cognitive_2016, anderson_neural_2010}, hierarchical reinforcement learning \cite{kolter_hierarchical_2007, dietterich_hierarchical_2000}, and compositionality research \cite{hudson_compositional_2018, hupkes_compositionality_2020, andreas_neural_2017, poggio2017and}. An orthogonal line of work explores direct execution of algorithms by neural networks \cite{velickovic_neural_2021, mahdavi_towards_2023, ibarz_generalist_2022, yan_neural_2020}. Unlike these approaches that focus on \emph{how} to integrate neural and symbolic components, our work addresses \emph{why} symbolic execution outperforms neural reasoning for algorithmic tasks.

\textbf{LLM Reasoning.} Recent work has explored various reasoning paradigms for LLMs, including symbolic reasoning \cite{marra_integrating_2019, olausson_linc_2023, han_folio_2024}, chain-of-thought prompting \cite{altabaa2025cot, zelikman_star_2022, merrill_expressive_2024, altabaa_cot_2025}, and in-context learning \cite{xie2021explanation, garg2022can, akyurek2022learning, zhang2024trained}. \citet{xie2021explanation} model in-context learning as implicit Bayesian inference, which we extend to compare different reasoning representations. While prior work demonstrates \emph{that} certain prompting strategies improve performance, we provide a theoretical framework explaining \emph{why} code representations lead to lower Bayes error.

\textbf{LLM Tool-Use.} Tool-augmented LLMs have achieved strong empirical results \cite{shen_llm_2024, schick_toolformer_nodate, qin_toolllm_2023, tang_toolalpaca_2023, parisi_talm_2022}. Code generation for tool-use can be viewed as a form of semantic parsing \cite{shin_few-shot_2022, krishnamurthy_neural_2017, berant_semantic_2013, dong_language_2016} or function calling \cite{puri_codenet_2021, alon_code2vec_2019, chen_neural_2018}. Our work complements this literature by providing theoretical justification for the observed empirical advantages of code-based tool-use over direct natural language reasoning. 

% \section{Discussion}

% \textbf{Summary.} This paper addresses whether algorithmic problems encoded in natural language should be solved via direct reasoning or by translating to code and executing with a solver. Our three-arm framework demonstrates that code execution consistently outperforms both code simulation and natural language reasoning, with theoretical backing from information theory.

% \textbf{Interpretation.} The theoretical analysis reveals that code representations yield higher mutual information with target algorithms than natural language representations. This explains the empirical observation: code serves as a more discriminative intermediate representation for implicit algorithm classification during LLM inference. The Bayesian framework makes this comparison tractable by decomposing the problem into translation and execution phases.

\section{Conclusion}
\label{sec:conclusion}
We introduced a three-route Bayesian framework (\cref{sec:framework}) for comparing code and natural language representations in algorithmic reasoning. By introducing an intermediate route---code generation with LLM-based simulation---we isolate translation effects from execution effects, enabling tractable pairwise analysis. Empirically, code execution achieves 4.01$\times$ higher odds of correctness than NL reasoning ($p<0.001$) across 44 algorithmic tasks and 6 models (\cref{sec:performance}). Our causal intervention experiments (\cref{sec:rep_analysis}) demonstrate that NL reasoning traces are distributionally and functionally equivalent to code-translated traces, supporting the hypothesis that NL reasoning implicitly simulates underlying algorithmic computations for specific models,tasks and prompts. Theoretically, we prove that code-based reasoning achieves lower Bayes risk via Blackwell dominance (\cref{prop:dominance}), providing an information-theoretic explanation for the empirical advantages of solver-based pipelines.

\paragraph{Limitations}

\paragraph{Future Work}

\section{Impact Statement}
This paper presents work whose goal is to advance the field of machine learning. There are many potential societal consequences of our work, none of which we feel must be specifically highlighted here.


% \textbf{Limitations.} (1)~\emph{Task scope}: Our evaluation focuses on algorithmic tasks (arithmetic, DP, ILP) with well-defined ground truth; results may not generalize to open-ended reasoning or tasks without clear algorithmic structure. (2)~\emph{Empirical approximations}: Our distributional similarity tests use finite samples and specific judge models, introducing estimation variance. (3)~\emph{Model coverage}: While we evaluate both frontier (GPT-4o, Claude, Gemini) and open-source models (Mistral, LLaMA), we cannot guarantee findings generalize to all architectures or future models. (4)~\emph{Theoretical assumptions}: The Blackwell dominance result assumes the existence of a garbling channel from code to NL, which we validate empirically but do not prove from first principles.






\bibliography{example_paper}
\bibliographystyle{icml2025}


%%%%%%%%%%%%%%%%%%%%%%%%%%%%%%%%%%%%%%%%%%%%%%%%%%%%%%%%%%%%%%%%%%%%%%%%%%%%%%%
%%%%%%%%%%%%%%%%%%%%%%%%%%%%%%%%%%%%%%%%%%%%%%%%%%%%%%%%%%%%%%%%%%%%%%%%%%%%%%%
% APPENDIX
%%%%%%%%%%%%%%%%%%%%%%%%%%%%%%%%%%%%%%%%%%%%%%%%%%%%%%%%%%%%%%%%%%%%%%%%%%%%%%%
%%%%%%%%%%%%%%%%%%%%%%%%%%%%%%%%%%%%%%%%%%%%%%%%%%%%%%%%%%%%%%%%%%%%%%%%%%%%%%%
\newpage
\appendix

% \section{Proof of Corollary 1} \label{app:cor1_proof}

% \begin{proof}
% Let $L_r = \max_\theta \mb E_{\gamma,z} \log q(\gamma| z,\theta, r)$ denote the optimal expected log-likelihood for representation $r$. Note that $H_{CE}(r) = -L_r$.

% From \cref{lem:2}, the mutual information is lower bounded by:
% \begin{align}
%     \mc I(\gamma, Z_r) \geq H(\gamma) + L_r.
% \end{align}

% Taking the difference between code and NL:
% \begin{align}
%     &\mc I(\gamma, Z_{\mathrm{Code}}) - \mc I(\gamma, Z_{\mathrm{NL}}) \nonumber \\
%     &\quad \geq [H(\gamma) + L_C] - [H(\gamma) + L_{NL}] \nonumber \\
%     &\quad = L_C - L_{NL} = H_{CE}(\mathrm{NL}) - H_{CE}(\mathrm{Code}).
% \end{align}

% When $H_{CE}(\mathrm{Code}) < H_{CE}(\mathrm{NL})$, equivalently $L_C > L_{NL}$, the RHS is positive:
% \begin{equation}
%     \mc I(\gamma, Z_{\mathrm{Code}}) > \mc I(\gamma, Z_{\mathrm{NL}}).
% \end{equation}

% Applying \cref{thm:1}:
% \begin{align}
%     P^*_{\mathrm{NL}} - P^*_{\mathrm{Code}} \geq \frac{\mc I(\gamma, Z_{\mathrm{Code}}) - \mc I(\gamma, Z_{\mathrm{NL}})}{\log_2(K-1)} > 0.
% \end{align}

% Therefore, $P^*_{\mathrm{Code}} < P^*_{\mathrm{NL}}$.
% \end{proof}

\end{document}


% This document was modified from the file originally made available by
% Pat Langley and Andrea Danyluk for ICML-2K. This version was created
% by Iain Murray in 2018, and modified by Alexandre Bouchard in
% 2019 and 2021 and by Csaba Szepesvari, Gang Niu and Sivan Sabato in 2022.
% Modified again in 2023 and 2024 by Sivan Sabato and Jonathan Scarlett.
% Previous contributors include Dan Roy, Lise Getoor and Tobias
% Scheffer, which was slightly modified from the 2010 version by
% Thorsten Joachims & Johannes Fuernkranz, slightly modified from the
% 2009 version by Kiri Wagstaff and Sam Roweis's 2008 version, which is
% slightly modified from Prasad Tadepalli's 2007 version which is a
% lightly changed version of the previous year's version by Andrew
% Moore, which was in turn edited from those of Kristian Kersting and
% Codrina Lauth. Alex Smola contributed to the algorithmic style files.
